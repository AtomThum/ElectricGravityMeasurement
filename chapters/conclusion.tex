\chapter{Conclusion}

In conclusion, we successfully demonstrated that the measurement of Earth's gravitational acceleration using a simple setup involving a free-falling magnet and a solenoid, analyzed through electromagnetic induction principles, is possible. The results indicated quite a strong correlation between the observed peak-to-peak voltage differences and the theoretical predictions, particularly aligning with the gravitational acceleration value of $\qty{9.529}{\meter\per\second^2}$, which is very close to Thailand's actual gravitational acceleration, which is around $\qty{9.78}{\meter\per\second^2}$.

However, it is important to note that the measured gravitational acceleration was slightly lower than the expected value. This discrepancy can be attributed to the experimental setup, specifically the lack of a guiding tube for the magnet. Without a tube, the magnet may have experienced slight collisions with the solenoid during its fall, which could have reduced the velocity of the magnet; thus, reducing the induced voltage also. Such interactions may have dampened the effectiveness of the electromagnetic induction process, leading to an underestimation of the gravitational force acting on the magnet.

Despite these challenges, the experiment validated the hypothesis that gravitational acceleration can be measured through induced voltage and highlighted the effectiveness of using basic materials and methods to achieve this goal. Future work could focus on refining the data processing methods and incorporating a guiding tube to minimize collisions, thereby enhancing the accuracy of gravitational measurements. Overall, this research underscores the potential of electromagnetic induction as a practical tool for measuring fundamental physical constants.

\newpage
